\chapter{game rules}
Below you can find the basic game rules which are implemented in the game logic. All rules are subject to change in further versions.

\section{how to play the game}
\begin{enumerate}
	\item 2 to n players, maximum number of players is 8.
	\item At the beginning of a game, the map is created randomly.
	\item You lose the game when your mother ship does not exist anymore.
	\item You win the game when your mother ship is the only remaining mother ship on the map.
	\item Starting positions of the players are assigned randomly, but in a way no player has an advantage or disadvantage.
	\item Every player is able to see the whole map.
	\item The map consists of stars. At the beginning of a game every player's mother ship is randomly assigned to a star.
	\item Each star has its own attributes, which influence:  
		\begin{enumerate}[label=\alph*)]
		\item The frequency of production of new regular ships.
		\item The basic defence of a star against intruders.
		\item How much effort it takes to colonize the star.
		\end{enumerate}
		\item If a player colonizes a new star, the star automatically starts to produce ships for the player, which are local to that star.
	\item The ships that are local to a star, that is owned by the player, can be sent to neighbouring stars
	\item Every command to move ships from one star to another originate from the mother ship.
	\item Each star also has an attribute that regulates the maximum radius a ship can reach without having to fuel up on another star.
	\item The rule above implicates	that further destinations can only be reached by one's regular ships by ``hopping'' from star to star to be able to fuel up.
	\item ``Hopping'' can occur on both neutral and owned (colonized) stars.
	\item By hopping onto a star that is owned by an opponent, the hopping player gets punished by dematerialization of a certain percentage of his hopping ships.
	\item If a player targets a certain star he or she can influence the hopping path his or her ships or use the one that is automatically calculated by the system, which is the shortest.
	\item Hopping and colonizing stars are two separate processes. Accidental colonization while hopping a star can not occur.
	\item When ships reach the target they hopped to, two scenarios can happen:
		\begin{enumerate}[label=\alph*)]
			\item Nobody is there yet and the star is considered neutral. The player's ships start colonizing the star. The speed of the colonization progress depends on the number of ships the player sent, the basic defence of the star and how much effort the star takes to be colonized.
			\item The star is already under possession of an opponent. In this case the fleets of the opponents start tearing down each other, until only regular ships of one of the fleets remain. If the remaining regular ships belong to the intruder, they immediately start neutralizing the star to then colonize it. If the remaining regular ships belong to the previous owner of the star, nothing else happens.
		\end{enumerate}
		\item Every move one's regular ships make originates from the mother ship. Thus, the order takes longer to get to the concerning regular ships, the further their base star is away from the mother ship.
	\item When a player orders a fleet to go colonize another star, the information about the number of available regular ships is always outdated (because of the distance). So the player has to options:
		\begin{enumerate}[label=\alph*)]
			\item He/She sends out the order that a certain percentage of the regular ships should leave their base star.
			\item He/She sends out the order that a certain number or regular ships should be the maximum number of sent out ships. If there are not that many regular ships left the moment the order arrives at the base, all regular ships are sent out.
		\end{enumerate}
	\item Fleets of regular ships that are hopping to their target send out information to their mother ship about their position in regular intervals.
	\item The mother ship can also be moved from her base, it's slower than the regular ships.
	\item The mother ship has its own basic defence value.
	\item If the base of the mother ship is intruded, the mother ship is the last one to be torn down.
	\item Every mother ship and every regular ship broadcast their position so that every player can see it.
\end{enumerate}